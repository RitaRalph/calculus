\documentclass{ximera}
\input{../../preamble.tex}
\author{Jim Talamo}
\license{Creative Commons 3.0 By-NC}
\outcome{Compute indefinite integrals of functions of linear functions without using a substitution.period at end of sentence}
\begin{document}
\begin{exercise}

Functions of linear functions arise frequently in this course as well as in others.  While an explicit substitution can be performed to calculate indefinite integrals of functions of linear functions, it is very helpful to be able to compute these quickly without carrying out the explicit substitution!  

Compute the following indefinite integrals WITHOUT carrying out an explicit substitution:

\begin{prompt} (Use $C$ for the constant of integration) \end{prompt}

\[\int \sin(2x) \d x = \begin{prompt}\answer{-\frac{1}{2}\cos(2x)+C}\end{prompt}\]

\[\int \frac{2}{3x+4} \d x = \begin{prompt}\answer{\frac{2}{3}\ln|3x+4|+C}\end{prompt}\]

\begin{hint}
Did you remember the absolute value?
\end{hint}

\[\int e^{\frac{x}{4}} \d x = \begin{prompt}\answer{4e^{\frac{x}{4}}+C}\end{prompt}\]

\[\int \frac{2}{\sqrt{4x+5}} \d x = \begin{prompt}\answer{\sqrt{4x+5}+C}\end{prompt}\]

\[\int \frac{1}{9x^2+4} \d x = \begin{prompt}\answer{\frac{1}{6} \arctan\left(\frac{3x}{2}\right)+C}\end{prompt}\]
\[\mbox{Hint: Write $9x^2$ as $(3x)^2$.}\]





\end{exercise}
\end{document}
