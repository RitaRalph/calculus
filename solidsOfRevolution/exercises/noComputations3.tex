\documentclass[handout]{ximera}
\input{../../preamble.tex}
\author{Jim Talamo}
\license{Creative Commons 3.0 By-NC}
\outcome{Think about Washer and Shell Method conceptually.}
\begin{document}


\begin{exercise}
 The region bounded by $x=\frac{2}{y}$, $x=4$ and $x=9$ is revolved about the $y$-axis.  If an integral or sum of integrals with respect to $y$ is used to compute the volume of the solid, which method should be used?
 
\begin{multipleChoice}
\choice[correct]{Washer Method}
\choice{Shell Method} 
\end{multipleChoice}

\end{exercise}


\begin{exercise}
 The region $R$ is bounded by $2x+y=3$, $y=0$, and $x=0$.  The Shell Method can be used to set up an integral with respect to $y$ that computes the resulting solid is $R$ is revolved about:
 
 
\begin{multipleChoice}
\choice{$x=2$}
\choice[correct]{$y=4$} 
\end{multipleChoice}

\end{exercise}

\begin{exercise}
 The region bounded by $y=e^x$, $x=0$ and $x=\ln(3)$ is revolved about the line $y=-3$.  Which method should be used to express the volume with a single integral?

\begin{multipleChoice}
\choice[correct]{Washer Method}
\choice{Shell Method} 
\choice{both methods require a single integral}
\choice{both methods require more than one integral}
\end{multipleChoice}

\end{exercise}

\begin{exercise}
 The region bounded by $y=x^2$ and $x=y^2$ is revolved about the line $x=3$.  If the Washer Method is used to calculate the volume or the resulting solid, we must:
 
\begin{multipleChoice}
\choice{integrate with respect to $x$.}
\choice[correct]{integrate with respect to $y$.} 
\end{multipleChoice}

\end{exercise}







\end{document}
