\documentclass{ximera}

\input{../../preamble.tex}

\author{Jim Talamo and Jason Miller}
\license{Creative Commons 3.0 By-NC}


\outcome{Use both Washer and Shell Method to set up a volume integral.}


\begin{document}
\begin{exercise}

The region $R$ lies in the first quadrant and is bounded by the curves $y=x$, $y=-x^{2}+6$ and $x=0$.  A solid is formed by revolving $R$ about the line $x=4$. 

To set up an integral or sum of integrals with respect to $x$ that would give the volume of the solid, which method should be used to find the volume?

\begin{multipleChoice}
\choice[correct]{Shell Method}
\choice{Washer Method}
\end{multipleChoice}

How many integrals with respect to $x$ will we need to express the volume of the solid? $\answer{1}$. 


\begin{exercise} 
The integral that gives the volume of the solid is: 
\[
V=\int_{\answer{0}}^{\answer{2}} \answer{ 2\pi \left( 4-x \right) \left(-x^{2}+6 -x \right)} \d x
\] 


\end{exercise}


To set up an integral or sum of integrals with respect to $y$ that would give the volume of the solid, which method should be used to find the volume?

\begin{multipleChoice}
\choice{Shell Method}
\choice[correct]{Washer Method}
\end{multipleChoice}

How many integrals with respect to $y$ will we need to express the volume of the solid? $\answer{2}$. 

\begin{exercise}

The sum of integrals that gives the volume of the solid is: 
\[
V= \int_{\answer{0}}^{\answer{2}} \answer{ \pi 16- \pi \left( 4-y \right)^{2} } \d y + \int_{\answer{2}}^{\answer{6}} \answer{ \pi 16 - \pi \left(4 -\sqrt{6-y} \right)^{2}} \d y
\]

\end{exercise}

\end{exercise}
\end{document}
