\documentclass{ximera}

\input{../../preamble.tex}

\author{Jim Talamo}
\license{Creative Commons 3.0 By-NC}


\outcome{Set up an integral that gives the area of a surface of revolution with respect to both $x$ and $y$.}
\outcome{Find  the area of a surface of revolution.}

\begin{document}
\begin{exercise}

The portion of the curve $y=\sin(2x)$ from $x=0$ to $x=\frac{\pi}{8}$ is revolved around the $y$-axis.

 \begin{image}
      \begin{tikzpicture}
        \begin{axis}[
            xmin=-.3, xmax=.5,
            domain=-1:1,
            ymin=-.3, ymax=1.2,
            clip=false,
            xtick = {-1,1,2},
            ytick = {-1,1,2,3},
            axis lines =center,
            xlabel=$y$, ylabel=$y$, every axis y label/.style={at=(current axis.above origin),anchor=south},
            every axis x label/.style={at=(current axis.right of origin),anchor=west},
            axis on top,
          ]
                              
         \addplot [penColor,thick,smooth,domain=0:.393]{sin(deg(2*x))};
          
         % ds and points
          	\addplot[color=penColor2,fill=penColor2,only marks,mark=*] coordinates{(.2,.389)};
		\addplot[color=penColor2,fill=penColor2,only marks,mark=*] coordinates{(.25,.479)};
		\addplot[ultra thick, penColor2] plot coordinates {(.2,.389) (.25,.479)};
          	\node[anchor=north, penColor2] at (axis cs:.3,.5) {$\Delta s$};
	%axis
	\addplot[ultra thick, penColor5, dotted] plot coordinates {(0,-.3) (0,1.2)};
	
          %r and point 
           	\addplot[thick, penColor2] plot coordinates {(0,.45) (.23,.45)};
		\node[anchor=north, penColor2] at (axis cs:.12,.55) {$r$};
          
          
        \addplot[color=penColor,fill=penColor,only marks,mark=*] coordinates{(0,0)};
	\addplot[color=penColor,fill=penColor,only marks,mark=*] coordinates{(.389,.7)};
          
          \node[penColor] at (axis cs:.2,.8) {$y=\sin(2x)$};
        \end{axis}
      \end{tikzpicture}
    \end{image}
 
 %%%%%%%Add rotated image
 
To set up an integral with respect to $x$ that gives the area of the surface of revolution, do the following:  

Since we have chosen to integrate with respect to $x$, we use the result:

\[ SA = \int_{x=a}^{x=b} 2 \pi r \d s\]

and we must express $r$ in terms of $x$ and $\d s$ in terms of $x$ and $\d x$.  


Let's start with $\d s$: 

\[
\d s = \sqrt{\answer{1+4\cos^2(2x)}} \d x
\]


\begin{exercise}
Note that is $r$ is the distance from the axis to the curve. This is a:

\begin{multipleChoice}
\choice{vertical distance}
\choice[correct]{horizontal distance}
\end{multipleChoice} 
Thus $r=x_{right}-x_{left}$.  

We note that the slice is at a location $(x,y)$, which happens to be on the curve.  This allows us to use the curve to express $y$ in terms of $x$ or $x$ in terms of $y$ if necessary.  

For the slice, the quantity $\sin(2x)$ expresses:
\begin{multipleChoice}
\choice[correct]{The $y$-value on the curve if the $x$-value of the slice is specified.}
\choice{The $x$-value of the slice.}
\end{multipleChoice} 

Since we have to express $r$ in terms of $x$, and we note that $x_{right}$ is on the curve, we must express it in terms of $x$.  Hence, $x_{right} = \answer{x}$.

Since $x_{left}$ lies on the $y$-axis, $x_{left} = \answer{0}$, and $r= \answer{x}$.

\end{exercise}

\begin{exercise}
Now we see that an integral that gives the surface area is: 
\[
SA= \int_{x=a}^{x=b} 2 \pi r \d s = \int_{x=\answer{0}}^{x=\answer{\pi/8}} \answer{2 \pi x} \sqrt{\answer{1+4\cos^2(2x)}} \d x
\]

\begin{exercise}
Using computational software of your choice, the integral to 3 decimal places shows that the surface area is $\answer{.957}$ square units.  
\end{exercise}

\end{exercise}

%%%%%%%%%%%%%%%%%

To set up an integral with respect to $y$ that gives the area of the surface of revolution, do the following:  

Since we have chosen to integrate with respect to $y$, we use the result:

\[ SA = \int_{x=a}^{x=b} 2 \pi r \d s\]

and we must express $r$ in terms of $y$ and $\d s$ in terms of $y$ and $\d y$.  


Let's start by describing the curve as a function of $y$.  Since $y=\sin(2x)$, we find:

\[
x= \answer{\frac{1}{2}\arcsin(y)}
\]

Now, let's find $\d s$.  Since we integrate with respect to $y$, we use $\d s = \sqrt{1+\left(\frac{\d x}{\d y}\right)^2} \d y$.

We first compute $\frac{\d x}{\d y} = \frac{1}{2} \answer{\frac{1}{\sqrt{1-y^2}}}$.  So: 

\[
\d s = \sqrt{\answer{1+\frac{1}{4-4y^2}}} \d y
\]


\begin{exercise}
Note that is $r$ is the distance from the axis to the curve. This is a:

\begin{multipleChoice}
\choice{vertical distance}
\choice[correct]{horizontal distance}
\end{multipleChoice} 
Thus $r=x_{right}-x_{left}$.  

We note that the slice is at a location $(x,y)$, which happens to be on the curve.  This allows us to use the curve to express $y$ in terms of $x$ or $x$ in terms of $y$ if necessary.  

For the slice, the quantity $\frac{1}{2} \arcsin(y)$ expresses:
\begin{multipleChoice}
\choice[correct]{The $x$-value on the curve if the $y$-value of the slice is specified.}
\choice{The $y$-value of the slice.}
\end{multipleChoice} 

Since we have to express $r$ in terms of $y$, and we note that $x_{right}$ is on the curve, we have $x_{right} = \answer{\frac{1}{2} \arcsin(y)}$.

Since $x_{left}$ lies on the $y$-axis, $x_{left} = \answer{0}$, and $r= \answer{\frac{1}{2} \arcsin(y)}$.

\end{exercise}

\begin{exercise}
Now we see that an integral that gives the surface area is: 
\[
SA= \int_{y=c}^{y=d} 2 \pi r \d s = \int_{y=\answer{0}}^{y=\answer{\sqrt{2}/2}} \answer{ \pi \arcsin(y)} \sqrt{\answer{1+\frac{1}{4-4y^2}}} \d y
\]

\begin{exercise}
Using computational software of your choice, the integral to 3 decimal places shows that the surface area is $\answer{.957}$ square units.  

Does this agree with he previous result?

\begin{multipleChoice}
\choice[correct]{Yes}
\choice{No}
\end{multipleChoice}

\end{exercise}


\end{exercise}
\end{exercise}
\end{document}
